%# --- Formats Chapter title and page headers
%% macro chapter_head(name, unnumbered=False)
\thispagestyle{empty}
\colorlet{shadecolor}{\chapterColor}
%% if unnumbered
\addcontentsline{toc}{chapter}{\VAR{name}}
\chapter*{\VAR{name}}
%% else
\chapter{\VAR{name}}
%% endif
%# thruthfully I have no fucking idea why the code below works. Found a solution after like 20 min of trial and error
\fancyhead{}
\lhead[\textcolor{\chapterColor}{\rule[-2pt]{\textwidth}{15pt}}]{\textcolor{\chapterColor}{\rule[-2pt]{\textwidth}{15pt}}\hspace{-\textwidth}\color{white}\hspace{4pt}\protect\thepage\hspace{1ex}-\hspace{1ex}\VAR{name}}
\rhead[\textcolor{\chapterColor}{\rule[-2pt]{\textwidth}{15pt}}\hspace{-\textwidth}\color{white}\VAR{name} \protect\thepage \hspace{4pt}]{\textcolor{\chapterColor}{\rule[-2pt]{\textwidth}{15pt}}}
\fancyhead[RO]{}
\fancyhead[RE]{\color{white}\VAR{name}\hspace{1ex}-\hspace{1ex}\protect\thepage \hspace{4pt}}
%% endmacro


%# --- loops through all images attached to item and checks to see if they should be placed here
%% macro resolve_images(item, loc='b')
%% for im in item.images
%% if im.loc == loc
\VAR{image(im, im.size)}
%% endif
%% endfor
%% endmacro


%# --- this is sort of a hacky work around that prevents the template from starting and ending multicols immediately 
%# --- if an area starts with a full width photo
%% macro resolve_images_area_head(item, loc='b', size='f')
%% if size in ['f', 'h']
%% for im in item.images
%% if im.loc == loc and im.size == size
\VAR{image(im, size='h')}
%% endif
%% endfor
%% else
%% for im in item.images
%% if im.loc == loc and im.size == size
\VAR{image(im, im.size)}
%% endif
%% endfor
%% endif
%% endmacro


%# --- Handles image placement. 
%# --- Skips placement of action photos if "skip_action_photos" format option is entered
%# --- Image placement syntax varies based on size
%% macro image(im, size='h')
%% if im.class_name == 'text_insert'
\VAR{text_insert(im)}
%% else
%% if ('skip_action_photos' in im.book.format_options) and (im.ref == 'pt') and ('force_include' not in im.format_options)
%# --- bad form, but its early and I couldn't figure out how to do this without using else
%% else
\phantomsection\label{\VAR{im.ref}:\VAR{im.item_id}}
%% if size in ['f', 'h']
	%% if size == 'f'
	\end{multicols}
	%% endif
	\setbox0=\hbox{\begin{overpic}[width=0.8\linewidth]{\VAR{im.path_o+im.out_file_name}}\VAR{caption1(im.latex_description)}
	\end{overpic}}
	\needspace{\ht0}
	\begin{center}
	\begin{overpic}[width=0.9\linewidth]{\VAR{im.path_o+im.out_file_name}}\VAR{caption1(im.latex_description)}
	\end{overpic}
	\end{center}
	%% if size == 'f'
	\begin{multicols}{2}
	%% endif
%% elif size == 'p'
	\includepdf[picturecommand*={\VAR{caption2(im.latex_description)}}]{\VAR{im.path_o+im.out_file_name}}
%% elif size == 'pr'
  \begin{landscape}
	\includepdf[angle=90, picturecommand*={\VAR{caption3(im.latex_description)}}]{\VAR{im.path_o+im.out_file_name}}
  \end{landscape}
%% elif size == 's'
    \clearpage
    \ifodd\value{page}\hbox{}\newpage\fi
	\includepdf[pages=-, picturecommand*={\VAR{caption2(im.latex_description)}}]{\VAR{im.path_o+im.out_file_name}}
%% endif
%% endif
%% endif
%% endmacro


%# --- Handles caption placement for half width (h) and full width (f) images. 
%% macro caption1(description)
%% if description
\put (0,5) {
\colorbox{\chapterColor}{
\parbox{0.7\linewidth}{
\textcolor{white}{
\VAR{description}
}}}}
%% endif
%% endmacro

%# --- Handles caption placement for page (p) and spread (s) images. 
%% macro caption2(description)
%% if description
\put (10,10) {
\colorbox{\chapterColor}{
\parbox{0.6\paperwidth}{
\textcolor{white}{
\VAR{description}
}}}}
%% endif
%% endmacro


%# --- Handles caption placement for page rotate (pr) images. 
%% macro caption3(description)
%% if description
\put (400,10) {
\rotatebox[origin =lb]{90}{
\colorbox{\chapterColor}{
\parbox{0.6\paperheight}{
\textcolor{white}{
\VAR{description}
}}}}}
%% endif
%% endmacro


%# --- Places a formatted title and description for the input climb. 
%% macro climbLabel(climb)
\needspace{2em}
\phantomsection\label{\VAR{climb.ref}:\VAR{climb.item_id}}
\colorbox{\VAR{climb.color_LaTeX}}{
\parbox{0.95\linewidth}{
\hspace{-1ex}\textbf{$\Box$
\VAR{climb.getRtNum()} \VAR{climb.name}\VAR{climb.name_unconfirmed_LaTeX} \VAR{climb.grade_str}\VAR{climb.grade_unconfirmed_LaTeX} \VAR{climb.rating_LaTeX} \VAR{climb.serious_LaTeX}
}}}
\begin{adjustwidth}{1.3em}{}			
\VAR{gps(climb)}
\VAR{climb.description}
%% if not climb.hasTopo
  (No Topo)
%% endif
\end{adjustwidth}
%% endmacro


%# --- Places a description, and bolded note so long as those attributes aren't empty
%% macro body(item)		
%% if item.description
\VAR{item.description}\\
%% endif
%% if item.note
\textbf{NOTE: \VAR{item.note}}\\
%% endif
%% endmacro


%# --- Checks if item has an associated gps qr code and places it if so
%% macro gps(item)
%% if item.gps and ('suppress_gps' not in item.format_options)
\VAR{qrcode(
  item.paths['qr_o']+'/'+item.item_id+'_qr.png',
  'http://maps.google.com/maps?q='+item.gps,
  'Navigate to this '+item.class_name
  )}
%% endif
%% endmacro


%# --- places a formatted qr code
%% macro qrcode(path, link, text)
\setbox0=\hbox{\includegraphics[width=0.45\linewidth]{\VAR{path}}}% Store image in \box0
\needspace{\ht0}% Need at least the height of \box0
\begin{center}
\includegraphics[width=0.45\linewidth]{\VAR{path}}
\end{center}
\begin{center}
\underline{\textcolor{blue}{\href{\VAR{link}}{\VAR{text}}}}
\end{center}
%% endmacro


%# --- Handles text insert placement. 
%# --- Skips placement if "skip_text_inserts" format option is entered
%% macro text_insert(ti)
%% if not (('skip_text_inserts' in ti.book.format_options) and ('force_include' not in ti.format_options))
\phantomsection\label{\VAR{ti.ref}:\VAR{ti.item_id}}
\vspace{1ex}
\colorbox{\chapterColor!40}{\parbox{0.95\linewidth}{
\large{\textbf{\VAR{ti.name}}}\\
\normalsize{\VAR{ti.description}}
}}
%% endif
%% endmacro