\documentclass[10pt]{report}
 \renewcommand{\familydefault}{\sfdefault}
\usepackage[a4paper,margin=1cm,bottom=0.5cm,top=1.5cm]{geometry}
\usepackage{enumitem}
\usepackage[explicit]{titlesec}
\usepackage{graphicx}
\usepackage{color}
\usepackage[dvipsnames]{xcolor}
\usepackage{lipsum}
\usepackage{pifont}
\usepackage{framed}
\usepackage{caption}
\usepackage{bookmark}
\usepackage{xcolor}
\usepackage{supertabular}
\usepackage{wrapfig}
\usepackage[section]{placeins}
\graphicspath{ {../} }


\usepackage{fancyhdr}
\pagestyle{fancy}

\setlength\parindent{0pt}
\setenumerate{leftmargin=*}
\renewcommand{\labelenumi}{\textbf{\arabic{enumi}}}
\renewcommand{\labelenumii}{\arabic{enumi}\alph{enumii}}

\setcounter{topnumber}{2}
\setcounter{bottomnumber}{2}
\setcounter{totalnumber}{4}

\titleformat{\chapter}[display]{\normalfont\color{white} \begin {shaded*}\bfseries}{\Huge#1}{0pt}{\end{shaded*}}
\titleformat{\section}[hang]{\normalfont\color{white}}{\colorbox{\chapterColor}{\parbox{\dimexpr\columnwidth-2\fboxsep}{\Large#1}}}{0pt}{}

\newcommand{\halfPic}[3]{
  \begin{figure}[h]
    \centering
      \captionsetup{singlelinecheck=false, labelformat=empty}
      \includegraphics[width=0.8\linewidth]{#2}
      \caption{\textbf{#3#1}}
  \end{figure}
}
\newcommand{\fullPic}[3]{
  \begin{figure*}[tbh]
    \centering
      \captionsetup{singlelinecheck=false, labelformat=empty}
      \includegraphics[width=0.8\textwidth]{#2}
      \caption{\textbf{#3#1}}
  \end{figure*}
}
\newcommand{\warn}{\ooalign{$\bigtriangleup$\cr\hidewidth!\hidewidth}}
\newcommand{\hlink}[2]{
\underline{\textcolor{blue}{\href{#1}{#2}}}
}
\newcommand{\qrcode}[3]{
  \begin{figure}[h]
    \centering
      \captionsetup{labelformat=empty}
      \includegraphics[width=0.45\linewidth]{#1}
      \caption{\hlink{#2}{#3}}
  \end{figure}
}


\begin{document}
\title{\VAR{book.name}}
\date{\VAR{book.date}}
\maketitle

\newcommand\chapterColor{MidnightBlue}
\twocolumn
\colorlet{shadecolor}{\chapterColor}
\chapter{Acknowledgements}
\markboth{\color{white}Acknowledgements \protect\thepage \hspace{4pt}}{}
\lhead{\textcolor{\chapterColor}{\rule[-2pt]{\textwidth}{15pt}}}
This guidebook was a collaborative effort built with the localBoulders framework. It is intended to be a living document if you notice any mistakes, errors, or omissions get in touch with the creators of this document or submit your own contribution via the book's \hlink{\VAR{book.repo}}{github repository}.
  \begin{figure}[h]
    \centering
      \captionsetup{labelformat=empty}
      \includegraphics[width=\linewidth]{./maps/qr/\VAR{book.name}_qr.png}
      \caption{\hlink{\VAR{book.dl}}{Get the latest revisionof this book}}
  \end{figure}
\subsection*{Contributors}
\begin{itemize}
%% for c in book.collaborators
\item \VAR{c}
%% endfor
\end{itemize}
\clearpage
\colorlet{shadecolor}{\chapterColor}
\chapter{Introduction}
\markboth{\color{white}Introduction \thepage}{}
\lhead{\textcolor{\chapterColor}{\rule[-2pt]{\textwidth}{15pt}}}
\lipsum[3]
\clearpage
\renewcommand\chapterColor{BrickRed}
%% for area in book.areas.values()
\input{areas/\VAR{area.name}}
\renewcommand\chapterColor{\VAR{loop.cycle('BurntOrange','PineGreen','RoyalPurple','Aquamarine','RubineRed','BrickRed')}}
%% endfor 
\renewcommand\chapterColor{MidnightBlue}
\input{index}

\end{document}